\documentclass[letterpaper,fleqn]{article}

\usepackage{amsmath}
\usepackage{amssymb}
\usepackage[letterpaper, margin=1in]{geometry}
\usepackage{xcolor, colortbl}
\usepackage{cancel}
\usepackage{tabularx}

\setlength{\extrarowheight}{3pt}

\begin{document}
	\newcommand{\set}[1]{Exercise Set 4.#1}
	
	\newcounter{pcount}
	\setcounter{pcount}{1}
	
	\newcommand{\problem}[2]{
%		\item []
		\noindent\begin{minipage}{\textwidth}
			\textbf{Problem \thepcount\stepcounter{pcount}.} \\
			Exercise Set 4.#1
			\begin{enumerate}
				#2
			\end{enumerate}
		\end{minipage} \\
		\\
	}
	
	\newcommand{\case}[2]{\cellcolor{yellow!25}\textbf{Case #1:} #2}
	\newcommand{\theorem}[2]{
		\cellcolor{green!25}\textbf{Theorem #1:} \\
		\hline
		#2
	}
	
%	\begin{enumerate}
		\problem{1}{
		\item [52.]
		For all integers $m$, if $m>2$ then $m^2-4$ is composite. \\
		False. \\
		\textbf{Counterexample:} \\
		Let $m=3$. \\
		$3^2-4=5$ \\
		5 is not composite.
		
		\item [53.]
		For all integers $n$, $n^2-n+11$ is a prime number. \\
		False. \\
		\textbf{Counterexample:} \\
		Let $n=11$. \\
		$11^2-11+11=121$ \\
		121 isn't prime.
		}
		
		\problem{1}{
		\item [61.]
		Suppose that integers $m$ and $n$ are perfect squares. Then $m+n+2\sqrt{mn}$ is also a perfect square. Why? \\
		\textbf{Proof:} \\
		Let $m=a^2$, where $a \in \mathbb{Z}$. \\
		Let $n=b^2$, where $b \in \mathbb{Z}$. \\
		$\begin{aligned}[t]
			m+n+2\sqrt{mn} ={} & a^2+b^2+2\sqrt{a^2b^2}  && \text{substitution} \\
			={} & a^2+b^2+2ab \\
			={} & (a+b)^2 && \text{factoring} \\
			={} & c^2 && \text{let $c=a+b$}
		\end{aligned}$ \\
		$c \in \mathbb{Z}$ because addition is closed under $\mathbb{Z}$ \\
		$m+n+2\sqrt{mn}$ is a perfect square by definition (square of some integer).
		}
		
		\problem{2}{
		\item [30.]
		Prove that if one solution for a quadratic equation of the form $x^2+bx+c=0$ is rational (where $b$ and $c$ are rational), then the other solution is also rational. \\
		\textbf{Proof:} \\
		Let $s$ and $r$ be the roots of the equation. \\
		$\begin{aligned}[t]
			x^2+bx+c ={} & (x-s)(x-r) && \text{factored form} \\
			={} & x^2-xr-xs+rs && \text{distribute} \\
			={} & x^2-(r+s)x+rs && \text{factor} \\
		\end{aligned}$ \\
		$b=-(r+s)$, $c=rs$ \\
		$b,\, c \in \mathbb{Q}$, therefore $-(r+s),\, rs \in \mathbb{Q}$. \\
		\begin{tabular}{|l|}
			\hline
			\case{1}{$r,s \in \mathbb{Q}$} \\
			\hline
			Proof done. \\
			\hline
			\case{2}{$r,s \not\in \mathbb{Q}$} \\
			\hline
			Irrelevant. \\
			\hline
			\case{3}{WLOG let $r \in \mathbb{Q}$ and $s \not\in \mathbb{Q}$.} \\
			\hline
			$-(r+s) \in \mathbb{Q} \Rightarrow r+s \in \mathbb{Q}$ \\
			By Theorem 4.5.3, $r+s \not\in \mathbb{Q}$. \\
			Contradiction. \\
			\hline
		\end{tabular} \\
		The only valid, relevant case shows that both $r$ and $s$ must be rational.
		}
		
		\problem{2}{
		\item [38.]
		The ``proof'' does not prove that $\dfrac{a}{b}+\dfrac{c}{d}$ is rational.
		
		\item [39.]
		The ``proof'' assumes $r+s \in \mathbb{Q}$ to prove $r+s \in \mathbb{Q}$.  
		}
		
		\problem{3}{
		\item [30.]
		For all integers $a$ and $n$, if $a \mid n^2$ and $a \leq n$ then $a \mid n$. \\
		False. \\
		\textbf{Counterexample:} \\
		Let $a=4$ and $n=6$. \\
		$4 \mid 36$ \\
		$4 \nmid 6$
		}
		
		\problem{3}{
		\item [34.]
		Is it possible to have 50 coins, made up of pennies, dimes, and quarters, that add up to \$3? Explain. \\
		Not possible. \\
		\textbf{Proof:} \\
		Let $p$, $d$, $q \in \mathbb{Z}^+$ be the number of pennies, dimes, and quarters, respectively. \\
		Assume there such a configuration is possible. \\
		$\begin{aligned}
			   &&    p+ &&    d+ &&    q & =50 \\
			   && .01p+ && .10d+ && .25q & =3 \\
			\\
			   &&    p+ &&  10d+ &&  25q & =300 \\
			   &&  -(p+ &&    d+ &&    q & =50) \\
			 = &&       &&   9d+ &&  24q & =250 \\
		\end{aligned}$ \\
		$3(3d+8q)=250$ \\
		$3d+8q=\dfrac{250}{3}$ \\
		$3d+8q \in \mathbb{Z}^+$ because addition and multiplication is closed on $\mathbb{Z}^+$, but $\dfrac{250}{3} \not\in \mathbb{Z}^+$. \\
		Contradiction, therefore no configuration exists.
		}
		
		\problem{3}{
		\item [35.]
		Two athletes run a circular track at a steady pace so that the first completes one round in 8 minutes and the second in 10 minutes. If they both start from the same spot at 4 \textsc{p.m.},	when will be the first time they return to the start together? \\
		At 4:40 \textsc{p.m.}
		}
		
		\problem{3}{
		\item [42.] \quad
		\begin{enumerate}
			\item [(c)] Without computing the value of $\left(20!\right)^2$ determine how many zeros are at the end of this number when it is written in decimal form. Justify your answer. \\
			8 zeroes.\\
			\textbf{Proof:} \\
			Multiplying by 10 adds a trailing zero. \\
			Factoring out all 10s from 20! gives the number of trailing zeros in 20!. \\
			$20!=1 \times 2 \times 3 \times 4 \times 5 \times 6 \times 7 \times 8 \times 9 \times 10 \times 11 \times 12 \times 13 \times 14 \times 15 \times 16 \times 17 \times 18 \times 19 \times 20$ \\
			$20!=\underbrace{(10 \times 20)}_{\text{\normalsize obvious 10s}} \times 1 \times 2 \times 3 \times 4 \times 5 \times 6 \times 7 \times 8 \times 9 \times 11 \times 12 \times 13 \times 14 \times 15 \times 16 \times 17 \times 18 \times 19$ \\
			$20!=\underbrace{(10 \times 20)}_{\text{\normalsize obvious 10s}}\overbrace{(2 \times 5)(4 \times 15)}^{\text{\normalsize remaining 10s}} \times 
			\underbrace{
			\left(\begin{aligned}
				1 \times 3 \times 4 \times 6 \times 7 \times 8 \times 9 \times 11 \times \\
				12 \times 13 \times 14 \times 16 \times 17 \times 18 \times 19
			\end{aligned}\right)}_{\text{\normalsize remaining factors}}$ \\
			Remaining factors don't have any more 10s to factor out and no more 5s to factor out to pair with 2s to create 10s. \\
			$20!=10^4 \times \overbrace{12 \times 
			\underbrace{
			\left(\begin{aligned}
					1 \times 3 \times 4 \times 6 \times 7 \times 8 \times 9 \times 11 \times \\
					12 \times 13 \times 14 \times 16 \times 17 \times 18 \times 19
			\end{aligned}\right)}_{\text{\normalsize remaining factors}}}^{a}=10^4a$ \\
			10 can be factored out of 20! 4 times, therefore 20! has 4 leading zeros. \\
			$\left(20!\right)^2=\left(10^4a\right)^2=10^8a^2$ \\
			$a$, like the remaining factors, doesn't have any 10s nor 5 and 2 pairs to factor out. \\
			10 can be factored out of $\left(20!\right)^2$ 8 times, therefore $\left(20!\right)^2$ has 8 trailing zeros.
		\end{enumerate}
		}
		
		\problem{3}{
		\item [43.]
		In a certain town 2/3 of the adult men are married to 3/5 of the adult women. Assume that all marriages are monogamous (no one is married to more than one other person). Also assume that there are at least 100 adult men in the town. What is the least possible number of adult men in the town? of adult women in the town? \\
		108 men and 120 women. \\
		\textbf{Proof:} \\
		Let $M \geq 100 \in \mathbb{Z}$ and $F \in \mathbb{N}$ be the number of men and women respectively. \\	
		Let $M_{married}=\dfrac{2}{3}M \in \mathbb{Z}$ and $F_{married}=\dfrac{3}{5}F \in \mathbb{Z}$ \\
		$M_{married}=F_{married} \Rightarrow \dfrac{2}{3}M = \dfrac{3}{5}F$ \\
		$F=\dfrac{10}{9}M$ and $M = \dfrac{9}{10}F$ \\
		No fractional number of people so $9 \mid M$ and $10 \mid F$ \\
		$F>M$ and we want to minimize adults, so find the minimal $M$. \\
		$M=$ ((Smallest multiple of $9)\geq100)=108$ \\
		$F=\dfrac{10}{9}M=\dfrac{10}{9}108=120$
		}
		
		\problem{4}{
		\item [30.] \quad
		\begin{enumerate}
			\item [(a)]
			Use the quotient-remainder theorem with $d=3$ to prove that the product of any two consecutive integers has the form $3k$ or $3k+2$ for some integer $k$. \\
			\textbf{Proof:} \\
			Let $n$ be an arbitrary integer. \\
			By the quotient remainder theorem, $n=3q+r$ where $q,\, r \in \mathbb{Z}$ and $0 \leq r < 3$. \\
			\begin{tabular}{|l|l|l|}
				\hline
				\case{1}{$n=3q+0$} &
				\case{2}{$n=3q+1$} &
				\case{3}{$n=3q+2$} \\
				\hline
				$n+1=3q+1$ & $n+1=3q+2$ & $n+1=3q+3$ \\
				$\begin{aligned}[t]
					n(n+1) &=3q(3q+1) \\
					&=3\underbrace{(9q^2+q)}_k
				\end{aligned}$ &
				$\begin{aligned}[t]
					n(n+1) &=(3q+1)(3q+2) \\
					&=9q^2+9q+2 \\
					&= 3\underbrace{(3q^2+3q)}_k+2
				\end{aligned}$ &
				$\begin{aligned}[t]
					n(n+1) &=(3q+2)(3q+3) \\
					&=(3q+2)(3(q+1)) \\
					&= 3\underbrace{(3q+2)(q+1)}_k
				\end{aligned}$ \\
				$n(n+1)=3k$ and $k \in \mathbb{Z}$ & 
				$n(n+1)=3k+2$ and $k \in \mathbb{Z}$ & 
				$n(n+1)=3k$ and $k \in \mathbb{Z}$ \\
				\hline
			\end{tabular} \\
			In all cases, the product of consecutive integers can be written in the form $3k$ or $3k+2$ for some integer $k$.
		\end{enumerate}
		}
		
		\problem{4}{
		\item [31.] \quad
		\begin{enumerate}
			\item [(a)]
			Prove that for all integers $m$ and $n$, $m+n$ and $m-n$ are either both odd or both even. \\
			\textbf{Proof:} \\
			\begin{tabular}{|l|}
				\hline
				\case{1}{ $m$ and $n$ are odd} \\
				\hline
				\\
				\begin{tabular}{|l|l|}
					\hline
					$\begin{aligned}
						m+n &= (2j+1)+(2k+1) \\
						&= 2j+2k+2 \\
						&= 2\underbrace{(j+k+1)}_a \\
						&= \text{$2a$ and $a \in \mathbb{Z}$ therefore even}
					\end{aligned}$ &
					$\begin{aligned}
						m-n &= (2j+1)-(2k+1) \\
						&= 2j-2k \\
						&= 2\underbrace{(j-k)}_a \\
						&= \text{$2a$ and $a \in \mathbb{Z}$ therefore even}
					\end{aligned}$ \\
					\hline
				\end{tabular} \\
				Both expressions are even. \\
				\hline
				\case{2}{$m$ and $n$ are even} \\
				\hline
				\\
				\begin{tabular}{|l|l|}
					\hline
					$\begin{aligned}
						m+n &= (2j)+(2k) \\
						&= 2\underbrace{(j+k)} \\
						&= \text{$2a$ and $a \in \mathbb{Z}$ therefore even}
					\end{aligned}$ &
					$\begin{aligned}
						m-n &= (2j)-(2k) \\
						&= 2\underbrace{(j-k)}_a \\
						&= \text{$2a$ and $a \in \mathbb{Z}$ therefore even}
					\end{aligned}$ \\
					\hline
				\end{tabular} \\
				Both expressions are even. \\
				\hline
				\case{3}{WLOG $m$ is odd and $n$ is even} \\
				\hline
				\\
				\begin{tabular}{|l|l|}
					\hline
					$\begin{aligned}
						m+n &= (2j+1)+(2k) \\
						&= 2j+2k+1 \\
						&= 2\underbrace{(j+k)}_a+1 \\
						&= \text{$2a+1$ and $a \in \mathbb{Z}$ therefore odd}
					\end{aligned}$ &
					$\begin{aligned}
						m-n &= (2j+1)-(2k) \\
						&= 2j-2k+1 \\
						&= 2\underbrace{(j-k)}_a+1 \\
						&= \text{$2a+1$ and $a \in \mathbb{Z}$ therefore odd}
					\end{aligned}$ \\
					\hline
				\end{tabular} \\
				Both expressions are odd. \\
				\hline
			\end{tabular} \\
			In all cases, $m+n$ and $m-n$ have the same parity.
		\end{enumerate}
		}
		
		\problem{4}{
		\item [40.]
		For any integer $n$, $n(n^2-1)(n+2)$ is divisible by 4. \\
		\textbf{Proof:} \\
		Let $E=n(n^2-1)(n+2)=n(n-1)(n+1)(n+2)=(n-1)(n)(n+1)(n+2)$. \\
		Let $a=n-1$. \\
		$E=a(a+1)(a+2)(a+3)$ \\
		\begin{tabular}{|l|l|}
			\hline
			\case{1}{$a$ mod $4=0$}  & \case{2}{$a$ mod $4=1$} \\
			\hline
			$a=4q$ where $q \in \mathbb{Z}$ & $a=4q+1$ where $q \in \mathbb{Z}$ \\
			$\begin{aligned}[t]
				E &=(4q)(4q+1)(4q+2)(4q+3) \\
				&=4\underbrace{q(4q+1)(4q+2)(4q+3)}_{k}
			\end{aligned}$ &
			$\begin{aligned}[t]
				E &=(4q+1)(4q+2)(4q+3)\overbrace{(4q+4)}^{4(q+1)} \\
				&=4\underbrace{(q+1)(4q+1)(4q+2)(4q+3)}_k \\
			\end{aligned}$
			\\
			$E=4k$ and k $\in \mathbb{Z}$ & $E=4k$ and k $\in \mathbb{Z}$ \\
			\hline
			\case{3}{$a$ mod $4=2$}  & \case{4}{$a$ mod $4=3$} \\
			\hline
			$a=4q$ where $q \in \mathbb{Z}$ & $a=4q+1$ where $q \in \mathbb{Z}$ \\
			$\begin{aligned}[t]
				E &=(4q+2)(4q+3)\overbrace{(4q+4)}^{4(q+1)}(4q+5) \\
				&=4\underbrace{(q+1)(4q+2)(4q+3)(4q+5)}_k \\
			\end{aligned}$ &
			$\begin{aligned}[t]
				E &=(4q+3)\overbrace{(4q+4)}^{4(q+1)}(4q+5)(4q+6) \\
				&=4\underbrace{(q+1)(4q+3)(4q+4)(4q+6)}_k \\
			\end{aligned}$
			\\
			$E=4k$ and k $\in \mathbb{Z}$ & $E=4k$ and k $\in \mathbb{Z}$ \\
			\hline
		\end{tabular} \\
		In all cases, $E=n(n^2-1)(n+2)$ is divisible by 4.
		}
		
		\problem{5}{
		\item [24.]
		The reciprocal of any irrational number is irrational. \\
		\textbf{Proof:} \\
		Using $\mathbb{I}$ as the set of irrationals. \\
		\begin{tabular}{|l|}
			\hline
			\theorem{A}{If $a \not= 0 \in \mathbb{Q}$ and $b \in \mathbb{I}$ then $ab \in \mathbb{I}$.} \\
			\hline
			\textbf{Proof:} \\
			Let $c=ab$. \\
			Assume opposite and $c \in \mathbb{Q}$. \\
			$ab=c \Rightarrow \dfrac{j}{k}b=\dfrac{m}{n}$ where $j$, $k$, $m$, $n \in \mathbb{Z}$ \\
			$b=\dfrac{mk}{nj}$ \\
			$mk,\, nj \in \mathbb{Z}$ \\
			$b$ is a ratio of two integers, therefore $b \in \mathbb{Q}$. \\
			Contradiction because $b \in \mathbb{I}$ and $b \in \mathbb{Q}$, therefore assumption is false and $b \in \mathbb{I}$. \\
			\hline
		\end{tabular} \\
		Let $a$ be an arbitrary irrational number. \\
		Assume opposite and $\dfrac{1}{a}$ is rational. \\
		$\dfrac{1}{a}=\dfrac{x}{y}$ where $x,\, y\not=0 \in \mathbb{Z}$ \\
		$y=ax$ \\
		By Theorem A, $ax=y$ must be irrational. \\
		Contradiction because $y \in \mathbb{Z}$ and $y \in \mathbb{I}$, therefore assumption is false and original proposition is true.
		}
		
		\problem{5}{
		\item [31.] \quad
		\begin{enumerate}
			\item [(b)]
			For all integers $n>1$, if $n$ is not prime, then there exists a prime number $p$ such that $p \leq \sqrt{n}$ and $n$ is divisible by $p$. \\
			\textbf{Proof:} \\
			\begin{tabular}{|l|}
				\hline
				\theorem{4.3.1}{For all integers $a$ and $b$, if $a$ and $b$ are positive and $a$ divides $b$, then $a < b$.} \\
				\hline
			\end{tabular} \\
			\begin{tabular}{|l|}
				\hline
				\theorem{4.3.3}{For all integers $a$, $b$, and $c$, if $a$ divides $b$ and $b$ divides $c$, then $a$ divides $c$.} \\
				\hline
			\end{tabular} \\
			\begin{tabular}{|l|}
				\hline
				\theorem{4.3.4}{Any integer $n>1$ is divisible by a prime number.} \\
				\hline
			\end{tabular} \\
			
			\begin{tabular}{|l|}
				\hline
				\theorem{A}{For any $n=ab$ where $a,\, b \in \mathbb{Z}^+$, if $a \leq b$ then $a \leq \sqrt{n}$ and $b \geq \sqrt{n}$.} \\
				\hline
				\textbf{Proof:} \\
				$n \in \mathbb{Z}^+ \Rightarrow \sqrt{n}\geq1$ \\
				Assume opposite. \\
				There exists $n=ab$ where $a,\, b \in \mathbb{Z}^+$ such that $a\leq b$ and ($a>\sqrt{n}$ or $b<\sqrt{n}$). \\
				By DeMorgan's laws, ($a\leq b$ and $a>\sqrt{n}$) or ($a\leq b$ and $b<\sqrt{n}$). \\
				\begin{tabular}{|l|}
					\hline
					\case{1}{$a\leq b$ and $a>\sqrt{n}$} \\
					\hline
					$\sqrt{n}<a\leq b$ \\
					$a>\sqrt{n}$ and $b>\sqrt{n}$ \\
					$b,\, \sqrt{n}>0$, therefore $ab>b\sqrt{n}$ and $b\sqrt{n}>n$. \\
					$ab>n$ \\
					Contradiction because $ab>n$ and $ab=n$. \\
					\hline
					\case{2}{$a\leq b$ and $b<\sqrt{n}$} \\
					\hline
					$a\leq b<\sqrt{n}$ \\
					$a<\sqrt{n}$ and $b<\sqrt{n}$ \\
					$b,\, \sqrt{n}>0$, therefore $ab<b\sqrt{n}$ and $b\sqrt{n}<n$. \\
					$ab<n$ \\
					Contradiction because $ab<n$ and $ab=n$. \\
					\hline
					\case{3}{($a\leq b$ and $a>\sqrt{n}$) and ($a\leq b$ and $b<\sqrt{n}$)} \\
					See previous cases. \\
					\hline
				\end{tabular} \\
				Contradiction in all cases of the opposite, therefore original proposition is true. \\
				\hline
			\end{tabular} \\
			
			By definition of composite number, $n=pq$ where $q \in \mathbb{Z}^+$. \\
			\begin{tabular}{|l|}
				\hline
				\case{1}{$p \leq q$} \\
				\hline
				By Theorem A, $p\leq\sqrt{n}$ and proposition is true. \\
				\hline
				\case{2}{$p>q$} \\
				\hline
				By Theorem A, $q\leq\sqrt{n}$. \\
				By Theorem 4.3.4, let $p_2$ be a prime factor of $q$. \\
				By Theorem 4.3.1, $p_2<q$. \\
				By Theorem 4.3.3, because $p_2 \mid q$ and $q \mid n$, $p_2 \mid n$. \\
				$p_2<q\leq\sqrt{n} \Rightarrow p_2\leq\sqrt{n}$ and proposition is true. \\
				\hline
			\end{tabular} \\
			In all cases the proposition is true.
		\end{enumerate}
		}
		
		\problem{5}{
		\item [33.]
		Use the sieve of Eratosthenes to find all prime numbers less than 100. \\
%		Cross out multiples up to $\sqrt{n=100}=10$. \\
		$\begin{array}{c c c c c c c c c c}
			& 2 & 3 & \textcolor{red}{\cancel{4}} & 5 & \textcolor{red}{\cancel{6}} & 7 & \textcolor{red}{\cancel{8}} & \textcolor{red}{\cancel{9}} & \textcolor{red}{\cancel{10}} \\
			11 & \textcolor{red}{\cancel{12}} & 13 & \textcolor{red}{\cancel{14}} & \textcolor{red}{\cancel{15}} & \textcolor{red}{\cancel{16}} & 17 & \textcolor{red}{\cancel{18}} & 19 & \textcolor{red}{\cancel{20}} \\
			\textcolor{red}{\cancel{21}} & \textcolor{red}{\cancel{22}} & 23 & \textcolor{red}{\cancel{24}} & \textcolor{red}{\cancel{25}} & \textcolor{red}{\cancel{26}} & \textcolor{red}{\cancel{27}} & \textcolor{red}{\cancel{28}} & 29 & \textcolor{red}{\cancel{30}} \\	
			31 & \textcolor{red}{\cancel{32}} & \textcolor{red}{\cancel{33}} & \textcolor{red}{\cancel{34}} & \textcolor{red}{\cancel{35}} & \textcolor{red}{\cancel{36}} & 37 & \textcolor{red}{\cancel{38}} & \textcolor{red}{\cancel{39}} & \textcolor{red}{\cancel{40}} \\
			41 & \textcolor{red}{\cancel{42}} & 43 & \textcolor{red}{\cancel{44}} & \textcolor{red}{\cancel{45}} & \textcolor{red}{\cancel{46}} & 47 & \textcolor{red}{\cancel{48}} & \textcolor{red}{\cancel{49}} & \textcolor{red}{\cancel{50}} \\
			\textcolor{red}{\cancel{51}} & \textcolor{red}{\cancel{52}} & 53 & \textcolor{red}{\cancel{54}} & \textcolor{red}{\cancel{55}} & \textcolor{red}{\cancel{56}} & \textcolor{red}{\cancel{57}} & \textcolor{red}{\cancel{58}} & 59 & \textcolor{red}{\cancel{60}} \\
			61 & \textcolor{red}{\cancel{62}} & \textcolor{red}{\cancel{63}} & \textcolor{red}{\cancel{64}} & \textcolor{red}{\cancel{65}} & \textcolor{red}{\cancel{66}} & 67 & \textcolor{red}{\cancel{68}} & \textcolor{red}{\cancel{69}} & \textcolor{red}{\cancel{70}} \\
			71 & \textcolor{red}{\cancel{72}} & 73 & \textcolor{red}{\cancel{74}} & \textcolor{red}{\cancel{75}} & \textcolor{red}{\cancel{76}} & \textcolor{red}{\cancel{77}} & \textcolor{red}{\cancel{78}} & 79 & \textcolor{red}{\cancel{80}} \\
			\textcolor{red}{\cancel{81}} & \textcolor{red}{\cancel{82}} & 83 & \textcolor{red}{\cancel{84}} & \textcolor{red}{\cancel{85}} & \textcolor{red}{\cancel{86}} & \textcolor{red}{\cancel{87}} & \textcolor{red}{\cancel{88}} & 89 & \textcolor{red}{\cancel{90}} \\
			\textcolor{red}{\cancel{91}} & \textcolor{red}{\cancel{92}} & \textcolor{red}{\cancel{93}} & \textcolor{red}{\cancel{94}} & \textcolor{red}{\cancel{95}} & \textcolor{red}{\cancel{96}} & 97 & \textcolor{red}{\cancel{98}} & \textcolor{red}{\cancel{99}} & \textcolor{red}{\cancel{100}} \\
		\end{array}$
		}
		
		\textbf{Problem 16.}
		\begin{itemize}
			\item [] Show that among any set of arbitrary (1 trillion + 1) natural numbers, one can find two numbers so that their difference is divisible by 1 trillion. \\
			\textbf{Proof:} \\
			Let $\mathbb{S} \subset \mathbb{N}$ be the set of $10^{12}+1$ arbitrary natural numbers. \\
			Map every element $e \in \mathbb{S}$ under $e$ \text{mod} $10^{12}$ and call the set $\mathbb{T}$. \\
			The maximum cardinality of $\mathbb{T}$ is $10^{12}$, because by quotient remainder theorem, remainders of division by $10^{12}$ are in the range $\left[0,10^{12}\right)$. The largest $\mathbb{T}$ possible contains all these remainders. \\
			Because $|\mathbb{S}|=10^{12}+1 > \text{max}(|\mathbb{T}|$)$=10^{12}$, by the pigeonhole principle, there exists two elements $a,\, b \in \mathbb{S}$ that map to the same element $m \in \mathbb{T}$. \\
			$m=a\text{ mod }10^{12} = b\text{ mod }10^{12}$ \\
			By the quotient remainder theorem, $a=10^{12}j+m$ and $b=10^{12}k+m$ where $j,\, k \in \mathbb{Z}$. \\
			$a-b = (10^{12}j+m)-(10^{12}k+m)=10^{12}(j-k)$ \\
			The difference $a-b$ is a multiple of $10^{12}$ therefore proposition is true. \\
		\end{itemize}
		
%	\end{enumerate}
\end{document}